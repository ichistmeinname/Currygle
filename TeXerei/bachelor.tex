% Header mit Deklarationen
\documentclass[%
	pdftex,%              PDFTex verwenden
	a4paper,%             A4 Papier
	oneside,%             Einseitig
	chapterprefix,%       Kapitel anschreiben als Kapitel
	headsepline,%         Linie nach Kopfzeile
	%footsepline,%         Linie vor Fusszeile
	%pointlessnumbers,%     Nummern ohne abschließenden Punkt
	12pt%                 Grössere Schrift, besser lesbar am bildschrim
]{scrbook}
%
% Paket für Übersetzungen ins Deutsche
%
% \usepackage[french,ngerman]{babel}

%
% Pakete um Latin1 Zeichnensätze verwenden zu können und die dazu
% passenden Schriften.
%
\usepackage[latin1]{inputenc}
\usepackage[T1]{fontenc}

%
% Paket für Quotes
%
% \usepackage[babel,french=guillemets,german=swiss]{csquotes}

%
% Paket zum Erweitern der Tabelleneigenschaften
%
\usepackage{array}

%
% Paket für schönere Tabellen
%
\usepackage{booktabs}

%
% Paket um Grafiken einbetten zu können
%
\usepackage{graphicx}

%
% Spezielle Schrift im Koma-Script setzen.
%
\usepackage{titlesec}
\newcommand*{\justifyheading}{\raggedleft}
\titleformat{\chapter}[display]
  {\normalfont\Huge\bfseries\justifyheading}{\thechapter}
  {10pt}{\Huge}
\titleformat{\section}
  {\normalfont\Large\bfseries}{\thesection}{1em}{}
\titleformat{\subsection}
  {\normalfont\large\bfseries}{\thesubsection}{1em}{}
% \setkomafont{sectioning}{\normalfont\bfseries}
\setkomafont{captionlabel}{\normalfont\bfseries} 
\setkomafont{pagehead}{\normalfont\bfseries} % Kopfzeilenschrift
\setkomafont{descriptionlabel}{\normalfont\bfseries}

%
% Zeilenumbruch bei Bildbeschreibungen.
%
\setcapindent{1em}

%
% Kopf und Fußzeilen
%
\usepackage{scrpage2}
\pagestyle{scrheadings}
% Inhalt bis Section rechts und Chapter links
\automark[section]{chapter}
% Mitte: leer
\chead{}

%
% mathematische symbole aus dem AMS Paket.
%
\usepackage{amsmath}
\usepackage{amssymb}

%
% Type 1 Fonts für bessere darstellung in PDF verwenden.
%
%\usepackage{mathptmx}           % Times + passende Mathefonts
%\usepackage[scaled=.92]{helvet} % skalierte Helvetica als \sfdefault
\usepackage{courier}            % Courier als \ttdefault

%
% Paket um Textteile drehen zu können
%
\usepackage{rotating}

%
% Paket für Farben im PDF
%
\usepackage{color}

%
% Paket für Links innerhalb des PDF Dokuments
%
\definecolor{LinkColor}{rgb}{0,0,0.5}
\usepackage[%
	pdftitle={An API Search Engine for Curry},% Titel der Diplomarbeit
	pdfauthor={Sandra Dylus},% Autor(en)
	pdfcreator={LaTeX, LaTeX with hyperref and KOMA-Script},% Genutzte Programme
	pdfsubject={Bachelor thesis}, % Betreff
	pdfkeywords={Curry, Haskell, API, search engine}]{hyperref} % Keywords halt :-)
\hypersetup{colorlinks=true,% Definition der Links im PDF File
	linkcolor=LinkColor,%
	citecolor=LinkColor,%
	filecolor=LinkColor,%
	menucolor=LinkColor,%
	pagecolor=LinkColor,%
	urlcolor=LinkColor}

%
% Paket um LIstings sauber zu formatieren.
%
\usepackage[savemem]{listings}
\lstloadlanguages{TeX}

%
% Listing Definationen für PHP Code
%
\definecolor{lbcolor}{rgb}{0.85,0.85,0.85}
\lstset{language=[LaTeX]TeX,
	numbers=left,
	stepnumber=1,
	numbersep=5pt,
	numberstyle=\tiny,
	breaklines=true,
	breakautoindent=true,
	postbreak=\space,
	tabsize=2,
	basicstyle=\ttfamily\footnotesize,
	showspaces=false,
	showstringspaces=false,
	extendedchars=true,
	backgroundcolor=\color{lbcolor}}
%
% ---------------------------------------------------------------------------
%

%
% Neue Umgebungen
%
\newenvironment{ListChanges}%
	{\begin{list}{$\diamondsuit$}{}}%
	{\end{list}}

%
% aller Bilder werden im Unterverzeichnis figures gesucht:
%
\graphicspath{{bilder/}}

%
% Literaturverzeichnis-Stil
%
\bibliographystyle{plain}

%
% Anführungsstriche mithilfe von \textss{-anzufuehrendes-}
%
\newcommand{\textss}[1]{"`#1"'}

%
% Strukturiertiefe bis subsubsection{} möglich
%
\setcounter{secnumdepth}{3}

%
% Dargestellte Strukturiertiefe im Inhaltsverzeichnis
%
\setcounter{tocdepth}{3}

%
% Zeilenabstand wird um den Faktor 1.5 verändert
%
%\renewcommand{\baselinestretch}{1.5}
%% ODER: format ==         = "\mathrel{==}"
%% ODER: format /=         = "\neq "
%
%
\makeatletter
\@ifundefined{lhs2tex.lhs2tex.sty.read}%
  {\@namedef{lhs2tex.lhs2tex.sty.read}{}%
   \newcommand\SkipToFmtEnd{}%
   \newcommand\EndFmtInput{}%
   \long\def\SkipToFmtEnd#1\EndFmtInput{}%
  }\SkipToFmtEnd

\newcommand\ReadOnlyOnce[1]{\@ifundefined{#1}{\@namedef{#1}{}}\SkipToFmtEnd}
\usepackage{amstext}
\usepackage{amssymb}
\usepackage{stmaryrd}
\DeclareFontFamily{OT1}{cmtex}{}
\DeclareFontShape{OT1}{cmtex}{m}{n}
  {<5><6><7><8>cmtex8
   <9>cmtex9
   <10><10.95><12><14.4><17.28><20.74><24.88>cmtex10}{}
\DeclareFontShape{OT1}{cmtex}{m}{it}
  {<-> ssub * cmtt/m/it}{}
\newcommand{\texfamily}{\fontfamily{cmtex}\selectfont}
\DeclareFontShape{OT1}{cmtt}{bx}{n}
  {<5><6><7><8>cmtt8
   <9>cmbtt9
   <10><10.95><12><14.4><17.28><20.74><24.88>cmbtt10}{}
\DeclareFontShape{OT1}{cmtex}{bx}{n}
  {<-> ssub * cmtt/bx/n}{}
\newcommand{\tex}[1]{\text{\texfamily#1}}	% NEU

\newcommand{\Sp}{\hskip.33334em\relax}


\newcommand{\Conid}[1]{\mathit{#1}}
\newcommand{\Varid}[1]{\mathit{#1}}
\newcommand{\anonymous}{\kern0.06em \vbox{\hrule\@width.5em}}
\newcommand{\plus}{\mathbin{+\!\!\!+}}
\newcommand{\bind}{\mathbin{>\!\!\!>\mkern-6.7mu=}}
\newcommand{\rbind}{\mathbin{=\mkern-6.7mu<\!\!\!<}}% suggested by Neil Mitchell
\newcommand{\sequ}{\mathbin{>\!\!\!>}}
\renewcommand{\leq}{\leqslant}
\renewcommand{\geq}{\geqslant}
\usepackage{polytable}

%mathindent has to be defined
\@ifundefined{mathindent}%
  {\newdimen\mathindent\mathindent\leftmargini}%
  {}%

\def\resethooks{%
  \global\let\SaveRestoreHook\empty
  \global\let\ColumnHook\empty}
\newcommand*{\savecolumns}[1][default]%
  {\g@addto@macro\SaveRestoreHook{\savecolumns[#1]}}
\newcommand*{\restorecolumns}[1][default]%
  {\g@addto@macro\SaveRestoreHook{\restorecolumns[#1]}}
\newcommand*{\aligncolumn}[2]%
  {\g@addto@macro\ColumnHook{\column{#1}{#2}}}

\resethooks

\newcommand{\onelinecommentchars}{\quad-{}- }
\newcommand{\commentbeginchars}{\enskip\{-}
\newcommand{\commentendchars}{-\}\enskip}

\newcommand{\visiblecomments}{%
  \let\onelinecomment=\onelinecommentchars
  \let\commentbegin=\commentbeginchars
  \let\commentend=\commentendchars}

\newcommand{\invisiblecomments}{%
  \let\onelinecomment=\empty
  \let\commentbegin=\empty
  \let\commentend=\empty}

\visiblecomments

\newlength{\blanklineskip}
\setlength{\blanklineskip}{0.66084ex}

\newcommand{\hsindent}[1]{\quad}% default is fixed indentation
\let\hspre\empty
\let\hspost\empty
\newcommand{\NB}{\textbf{NB}}
\newcommand{\Todo}[1]{$\langle$\textbf{To do:}~#1$\rangle$}

\EndFmtInput
\makeatother
%
%
%
%
%
%
% This package provides two environments suitable to take the place
% of hscode, called "plainhscode" and "arrayhscode". 
%
% The plain environment surrounds each code block by vertical space,
% and it uses \abovedisplayskip and \belowdisplayskip to get spacing
% similar to formulas. Note that if these dimensions are changed,
% the spacing around displayed math formulas changes as well.
% All code is indented using \leftskip.
%
% Changed 19.08.2004 to reflect changes in colorcode. Should work with
% CodeGroup.sty.
%
\ReadOnlyOnce{polycode.fmt}%
\makeatletter

\newcommand{\hsnewpar}[1]%
  {{\parskip=0pt\parindent=0pt\par\vskip #1\noindent}}

% can be used, for instance, to redefine the code size, by setting the
% command to \small or something alike
\newcommand{\hscodestyle}{}

% The command \sethscode can be used to switch the code formatting
% behaviour by mapping the hscode environment in the subst directive
% to a new LaTeX environment.

\newcommand{\sethscode}[1]%
  {\expandafter\let\expandafter\hscode\csname #1\endcsname
   \expandafter\let\expandafter\endhscode\csname end#1\endcsname}

% "compatibility" mode restores the non-polycode.fmt layout.

\newenvironment{compathscode}%
  {\par\noindent
   \advance\leftskip\mathindent
   \hscodestyle
   \let\\=\@normalcr
   \let\hspre\(\let\hspost\)%
   \pboxed}%
  {\endpboxed\)%
   \par\noindent
   \ignorespacesafterend}

\newcommand{\compaths}{\sethscode{compathscode}}

% "plain" mode is the proposed default.
% It should now work with \centering.
% This required some changes. The old version
% is still available for reference as oldplainhscode.

\newenvironment{plainhscode}%
  {\hsnewpar\abovedisplayskip
   \advance\leftskip\mathindent
   \hscodestyle
   \let\hspre\(\let\hspost\)%
   \pboxed}%
  {\endpboxed%
   \hsnewpar\belowdisplayskip
   \ignorespacesafterend}

\newenvironment{oldplainhscode}%
  {\hsnewpar\abovedisplayskip
   \advance\leftskip\mathindent
   \hscodestyle
   \let\\=\@normalcr
   \(\pboxed}%
  {\endpboxed\)%
   \hsnewpar\belowdisplayskip
   \ignorespacesafterend}

% Here, we make plainhscode the default environment.

\newcommand{\plainhs}{\sethscode{plainhscode}}
\newcommand{\oldplainhs}{\sethscode{oldplainhscode}}
\plainhs

% The arrayhscode is like plain, but makes use of polytable's
% parray environment which disallows page breaks in code blocks.

\newenvironment{arrayhscode}%
  {\hsnewpar\abovedisplayskip
   \advance\leftskip\mathindent
   \hscodestyle
   \let\\=\@normalcr
   \(\parray}%
  {\endparray\)%
   \hsnewpar\belowdisplayskip
   \ignorespacesafterend}

\newcommand{\arrayhs}{\sethscode{arrayhscode}}

% The mathhscode environment also makes use of polytable's parray 
% environment. It is supposed to be used only inside math mode 
% (I used it to typeset the type rules in my thesis).

\newenvironment{mathhscode}%
  {\parray}{\endparray}

\newcommand{\mathhs}{\sethscode{mathhscode}}

% texths is similar to mathhs, but works in text mode.

\newenvironment{texthscode}%
  {\(\parray}{\endparray\)}

\newcommand{\texths}{\sethscode{texthscode}}

% The framed environment places code in a framed box.

\def\codeframewidth{\arrayrulewidth}
\RequirePackage{calc}

\newenvironment{framedhscode}%
  {\parskip=\abovedisplayskip\par\noindent
   \hscodestyle
   \arrayrulewidth=\codeframewidth
   \tabular{@{}|p{\linewidth-2\arraycolsep-2\arrayrulewidth-2pt}|@{}}%
   \hline\framedhslinecorrect\\{-1.5ex}%
   \let\endoflinesave=\\
   \let\\=\@normalcr
   \(\pboxed}%
  {\endpboxed\)%
   \framedhslinecorrect\endoflinesave{.5ex}\hline
   \endtabular
   \parskip=\belowdisplayskip\par\noindent
   \ignorespacesafterend}

\newcommand{\framedhslinecorrect}[2]%
  {#1[#2]}

\newcommand{\framedhs}{\sethscode{framedhscode}}

% The inlinehscode environment is an experimental environment
% that can be used to typeset displayed code inline.

\newenvironment{inlinehscode}%
  {\(\def\column##1##2{}%
   \let\>\undefined\let\<\undefined\let\\\undefined
   \newcommand\>[1][]{}\newcommand\<[1][]{}\newcommand\\[1][]{}%
   \def\fromto##1##2##3{##3}%
   \def\nextline{}}{\) }%

\newcommand{\inlinehs}{\sethscode{inlinehscode}}

% The joincode environment is a separate environment that
% can be used to surround and thereby connect multiple code
% blocks.

\newenvironment{joincode}%
  {\let\orighscode=\hscode
   \let\origendhscode=\endhscode
   \def\endhscode{\def\hscode{\endgroup\def\@currenvir{hscode}\\}\begingroup}
   %\let\SaveRestoreHook=\empty
   %\let\ColumnHook=\empty
   %\let\resethooks=\empty
   \orighscode\def\hscode{\endgroup\def\@currenvir{hscode}}}%
  {\origendhscode
   \global\let\hscode=\orighscode
   \global\let\endhscode=\origendhscode}%

\makeatother
\EndFmtInput
%



\begin{document}

% Römische Nummerierung für Sonderseiten, wie Verzeichnisse und Anhang
\pagenumbering{Roman}

% Titelblatt
% Die Titelseite
% Im folgenden kommen ein paar Variablen, die auszufüllen sind
% Bisher steht dort nur Musterinhalt
% Außerdem müssen zei Dateien erstellt werden, Bild/Logo/Emblem des Fachgebietes
% sowie der Universität

\newcommand{\trtitle}{An API Search Engine for Curry}
\newcommand{\trtype}{Bachelor Thesis}
\newcommand{\trauthor}{Sandra Dylus}
\newcommand{\trstrasse}{Gutenbergstra\ss e 3}
\newcommand{\trmatrikelnummer}{1083}
\newcommand{\trort}{24116 Kiel}
\newcommand{\trbetreuer}{M.o.Sc. Bj\"orn Peem\"oller}
\newcommand{\trprof}{Prof. Dr. Michael Hanus}
\newcommand{\trfachgebiet}{Programming Languages and Compiler Construction}
\newcommand{\trinstitut}{Computer Science}
\newcommand{\trfakultaet}{Technische Fakult\"at}
\newcommand{\truni}{Christian-Albrechts-Universit\"at zu Kiel}
\newcommand{\trdate}{\today}

\thispagestyle{empty}

% Kopfzeile mit Logos.
% Eventuell die \hspace{} je nach Logogröße anpassen
\begin{tabular}{lcr}
  % \includegraphics[scale=0.8]{dein_unilogo} & % dein_unilogo.jpg/.eps im Verzeichnis "bilder" ablegen
  % \hspace{2cm}  
\truni 
% \hspace{2cm} &
  % \includegraphics[scale=0.8]{dein_fglogo} % dein_fglogo.jpg/.eps im Verzeichnis "bilder" ablegen, Fachgebietslogo
  \\
\end{tabular}

\rule{\textwidth}{0.4pt}

\vspace{2.5cm}
\begin{center}
  \textbf{\LARGE \trtitle}
\end{center}
\vspace{2cm}

\begin{center}
  \textbf{\trtype} \\
  \trfachgebiet \\
  \trprof \\
  Department for \trinstitut \\
  % \trfakultaet \\
  \truni \\[0.5cm]
  % vorgelegt von \\
  \textbf{\trauthor}
\end{center}

\vspace{1cm}


\begin{center}
\begin{tabular}{ll}
Advised by & \trbetreuer \\
\end{tabular}
\end{center}

\vfill

\begin{tabular}{l}
\trauthor \\
Matrikelnummer:  \trmatrikelnummer \\
\trstrasse \\
\trort
\end{tabular}

\rule{\textwidth}{0.4pt}

% Verzeichnisse
% Kopfzeile links Kapitel, rechts leer
\renewcommand{\chaptermark}[1]{\markboth{\thechapter\ #1}{}}
\ihead{\leftmark}
\ohead{}
%
% Inhaltsverzeichnis
%
\tableofcontents

%
% Abbildungsverzeichnis
%
\listoffigures

%
% Tabellenverzeichnis
%
\listoftables

% Merke mir die römische Seitenzahl in 'roemisch' und setzte Nummeriernung 
% auf arabisch für die eigentlichen Kapitel
\newpage
\newcounter{roemisch}
\setcounter{roemisch}{\value{page}}
\pagenumbering{arabic}

% Die einzelnen Kapitel
% Kopfzeile: links Kapitel, rechts Sektion
\ihead{\leftmark}
\ohead{\rightmark}
\chapter{Introduction}
Once upon a time ...
\section{Motivation}
Curr(y)gle > Hayoo!
\section{Structure}
At first \hyperref[preliminaries]{Chapter \ref*{preliminaries}} gives
the preliminiaries to understand this thesis. This includes basic
information about the programming language Curry, a tool to generate
documentation, and the Holumbus framework that is used to build the
search engine in connection with this thesis. In
\hyperref[analysis]{Chapter \ref*{analysis}} we analyze the
requirements to create an API search engine for Curry. It outlines the
first ideas for the following implementation, which is given in
\hyperref[implementation]{Chapter \ref*{implementation}}. Thereby we
focus on the most important implementation ideas and decisions. At the
very end we consider the results of this development. Also we give a
short outlook on features or ideas to expand the given result.
\chapter{Preliminaries}\label{preliminaries}
This chapter gives a brief introduction to background information
that are necessary to comprehend the following chapters. The first
section gives an introduction to the programming language Curry. It
outlines main concepts and features of the language and gives short
explanations for a better understanding. The Curry implementation we
refer to in the following sections is PAKCS\cite{pakcs}. Furthermore
we present CurryDoc\cite{currydoc}, a tool to generate documentation
that is distributed with PAKCS, in the second section. The last
section introduces the Holumbus\cite{holumbus} framework, a library
written in Haskell to configure and build search engines.

\section{The programming language Curry}\label{preliminaries:curry}

% Say that curry is a functional logic programming language and what
% this section is about.\\

Curry is a functional logic programming language, that is an
international development project to provide a platform for research
and teaching mostly. As the description suggest, it offers features of
both programming paradigms. The first subsection will start with some general
features, followed by two subsections that cover
functional and logical concepts of Curry.

% Start with the structure of a curry program. \\
\subsection{General overview}
Like in Haskell, a program consists of function definitions and data
structures. A Curry module \emph{Test} is a program that is saved as
\emph{Test.curry}.
% After that explain how a function definition looks like and how
% pattern matching works (mention left-hand and right-hand
% side). Currys layout applies to the so called off-side rule. Add
% that Curry is strongly-typed and also type-inferred, so a signature
% is optional. \\
The syntax of a program is quite similar to Haskell, where function
application are also written in juxtaposition.
\begin{hscode}\SaveRestoreHook
\column{B}{@{}>{\hspre}l<{\hspost}@{}}%
\column{E}{@{}>{\hspre}l<{\hspost}@{}}%
\>[B]{}\Varid{addTwo}\;\Varid{x}\mathrel{=}\Varid{x}\mathbin{+}\mathrm{2}{}\<[E]%
\ColumnHook
\end{hscode}\resethooks
The left-hand side of this function \ensuremath{\Varid{addTwo}\;\Varid{x}} is evaluated to the
right-hand side \ensuremath{\Varid{x}\mathbin{+}\mathrm{2}}, i.e. the call \ensuremath{\Varid{addTwo}\;\mathrm{3}} yields \ensuremath{\mathrm{3}\mathbin{+}\mathrm{3}\mathrel{=}\mathrm{5}}. Curry supports function definition with pattern matching, which is
often used in functional and logical programming languages. That means
that instantiated values like \ensuremath{\Conid{True}} and \ensuremath{\Conid{False}} can stand on the
left-hand side of a definition. The boolean operation \emph{not} is a
good example for a definition with pattern matching, since the result
depends on the input. The definition distinguishes between more than
one value , so we have to write one rule for each possible input
value.
\begin{hscode}\SaveRestoreHook
\column{B}{@{}>{\hspre}l<{\hspost}@{}}%
\column{E}{@{}>{\hspre}l<{\hspost}@{}}%
\>[B]{}\neg \mathbin{::}\Conid{Bool}\to \Conid{Bool}{}\<[E]%
\\
\>[B]{}\neg \;\Conid{False}\mathrel{=}\Conid{True}{}\<[E]%
\\
\>[B]{}\neg \;\Conid{True}\mathrel{=}\Conid{False}{}\<[E]%
\ColumnHook
\end{hscode}\resethooks
The first line indicates that the function expects a boolean value as
argument and yields to a boolean value as well. The following two
lines are rules, that describe that \ensuremath{\neg \;\Conid{False}} yields \ensuremath{\Conid{True}}, whereas
\ensuremath{\neg \;\Conid{True}} yields \ensuremath{\Conid{False}}. There are no more possible values for the
argument of the function \ensuremath{\neg }, since Curry is a strongly-typed
language and \ensuremath{\Conid{True}} and \ensuremath{\Conid{False}} are the only possible values with
boolean type.  In addition Curry also allows polymorphism. You can use
polymorphism for functions that work independent of the value's
type. For instance the identity function just returns the argument
that you apply to the function. This means you can apply the function
to all types of values, because the type does not matter. You use a
type variable in your type signature to indicate a polymorphic
type. The following code presents the type signature of the identity
function.
\begin{hscode}\SaveRestoreHook
\column{B}{@{}>{\hspre}l<{\hspost}@{}}%
\column{E}{@{}>{\hspre}l<{\hspost}@{}}%
\>[B]{}\Varid{id}\mathbin{::}\Varid{a}\to \Varid{a}{}\<[E]%
\\
\>[B]{}\Varid{id}\;\Varid{value}\mathrel{=}\Varid{value}{}\<[E]%
\ColumnHook
\end{hscode}\resethooks

Overall a strong type system as well as polymorphism and pattern
matching are among Curry's general features. The
next section covers functional characteristics of Curry.
% Mention that there are more concepts provided by Curry, which can be
% grouped by their underlying paradigm (use it as connection to the
% following subsections). Also the evaluation models are introduced.\\

% Curry is a functional logic programming language. As the name suggests
% it offers features of both programming paradigms. This section will
% outline the basic characteristics that Curry uses and give a short
% introduction to these characteristics.

% A program written in Curry is a collection of function definitions and
% data structures, which can be organized as a module. A module with the
% name \emph{M} is saved as \emph{M.curry} and can be imported in other
% modules. 
% The syntax of function definitions is quite similar to
% Haskell. Most functions are used as prefix operators, so the
% application of a function to an expression is denoted by
% juxtaposition. Arithmetic operators like |+| or |-| are infix
% operators like in the mathematical notation. Furthermore Curry also
% supports a conditional expression that corresponds to
% \emph{if-then-else}-expressions of other programming languages.

% In general one can say that Curry combines the features of the
% functional programming language Haskell and the logic programming
% language Prolog.

% \begin{itemize}
% \item pattern matching
% \item types (success)
% \item strongly typed
% \item type inference algorithm
% \item (pakcs) lazy, but incomplete - the strategy evaluates
%   non-deterministic choices sequentially instead of concurrently
% \item statically scoped (depends on layout - also: layout/off-side
%   rule), local scope with where and let
% \end{itemize}
\subsection{Functional features}\label{preliminaries:curryInfo}
% Begin with data structures that als can be defined by the user. Use
% CurryInfo as example.\\
The boolean values mentioned in the previous section are part of the
predefined data structures.
\begin{hscode}\SaveRestoreHook
\column{B}{@{}>{\hspre}l<{\hspost}@{}}%
\column{E}{@{}>{\hspre}l<{\hspost}@{}}%
\>[B]{}\mathbf{data}\;\Conid{Bool}\mathrel{=}\Conid{True}\mid \Conid{False}{}\<[E]%
\ColumnHook
\end{hscode}\resethooks
This code defines a data structure with the name \emph{Bool} that has
to constructors \ensuremath{\Conid{False}} and \ensuremath{\Conid{True}} with the type \ensuremath{\Conid{Bool}}. In this case
the constructors are unary. Let's define a data structure that is more
interesting and will be used in the following chapters. For example a
data structure that represents a Curry program, because we want to
create an API search engine for Curry. You can save a Curry program as
a module and it exists of functions and types (data structures).
\begin{hscode}\SaveRestoreHook
\column{B}{@{}>{\hspre}l<{\hspost}@{}}%
\column{E}{@{}>{\hspre}l<{\hspost}@{}}%
\>[B]{}\mathbf{data}\;\Conid{CurryInfo}\mathrel{=}\Conid{CurryInfo}\;\Conid{ModuleInfo}\;[\mskip1.5mu \Conid{FunctionInfo}\mskip1.5mu]\;[\mskip1.5mu \Conid{TypeInfo}\mskip1.5mu]{}\<[E]%
\ColumnHook
\end{hscode}\resethooks
In this example \emph{CurryInfo} is the name of the type and also the
name of the constructor. As a constructor it needs three arguments, a
\ensuremath{\Conid{ModuleInfo}} and a list of \ensuremath{\Conid{FunctionInfo}} and \ensuremath{\Conid{TypeInfo}}. We define
these data structures that hold the information about specific
proporties of the program in chapter 3.

% Introduce the idea of higher-order functions.\\
In Curry functions are first-class citizens. This means that they can
appear as argument of an expression or even in a data structure. The
most popular use-case is the manipulation of all elements of a list by
a given function. PAKCS provides the function map, which also exists
in Haskell, that takes two arguments, a function and a list and
returns a list. The important thing is, that the type of the function
matches the elements of the list. For a example a function, that
converts an integer to a character can be applied to a list of
integers and yields to list of characters. The following code presents
the a definition of the function map.
\begin{hscode}\SaveRestoreHook
\column{B}{@{}>{\hspre}l<{\hspost}@{}}%
\column{E}{@{}>{\hspre}l<{\hspost}@{}}%
\>[B]{}\Varid{map}\mathbin{::}(\Varid{a}\to \Varid{b})\to [\mskip1.5mu \Varid{a}\mskip1.5mu]\to [\mskip1.5mu \Varid{b}\mskip1.5mu]{}\<[E]%
\\
\>[B]{}\Varid{map}\;\Varid{function}\;[\mskip1.5mu \mskip1.5mu]\mathrel{=}[\mskip1.5mu \mskip1.5mu]{}\<[E]%
\\
\>[B]{}\Varid{map}\;\Varid{function}\;(\Varid{element}\mathbin{:}\Varid{list})\mathrel{=}\Varid{function}\;\Varid{element}\mathbin{:}\Varid{map}\;\Varid{list}{}\<[E]%
\ColumnHook
\end{hscode}\resethooks
The syntax for lists is the same as in Haskell. \ensuremath{[\mskip1.5mu \mskip1.5mu]} is the empty
list, whereas \ensuremath{\mathrm{1}\mathbin{:}\mathrm{2}\mathbin{:}\mathrm{3}\mathbin{:}[\mskip1.5mu \mskip1.5mu]} is the same as \ensuremath{[\mskip1.5mu \mathrm{1},\mathrm{2},\mathrm{3}\mskip1.5mu]} or \ensuremath{\mathrm{1}\mathbin{:}[\mskip1.5mu \mathrm{2},\mathrm{3}\mskip1.5mu]}. The
first representation is often used in pattern matching, to use the
first element of the list in the right-hand side of the
definition. 
The first line presents the type signature. The function
\ensuremath{(\Varid{a}\to \Varid{b})} takes a value of an unspecific type and returns another
unspecific type. The second argument is a list of elements with the
type, that the function expects. Furthermore the resulting list
contains elements that have the same type as the resulting type of the
function. 
The definition of map says that an empty list yields an
empty list, whereas the function is applied to
the elements of a list with at least one element recursively.\\

% Explain lazy evaluation. Mention infinite structures.
The last functional concept Curry supports is \emph{lazy
  evaluation}. In general an expression is evaluated by replacing the
left-hand side of a definition by the right-hand side. The evaluation
proceeds one replacement after another until it yields a value. A
value is an expression that only consists of built-in data structures
or literals. If the last replacement does not result in a value, the
evaluation fails. If an evaluation has more than one possible
replacement step, so called subexpression can be evaluated. Lazy
evaluation means that such a subexpression is only evaluated, if its
result is necessary to continue the evaluation.\\

In summary Curry's functional programming features covers data
structures, higher-order functions and lazy evaluation as evaluation
strategy. Since the logic characteristics of Curry induce that
there is another strategy besides lazy evaluation, we will go one to
the next section and take a closer look into these characteristics.
\subsection{Logic features}
% Start with the connection between functional and logical programming:
% introduce the logical feature of non-deterministic functions. Therefor
% give an example for a rule and later a non-deterministic function.\\

Besides the already mentioned functional characteristics, Curry also
offers non-deterministic functions and logical variables.
Logic programming languages consist of rules, for example a constant
function that represents my favourite number.
\begin{hscode}\SaveRestoreHook
\column{B}{@{}>{\hspre}l<{\hspost}@{}}%
\column{E}{@{}>{\hspre}l<{\hspost}@{}}%
\>[B]{}\Varid{favouriteNumber}\;\Conid{Sandra}\mathrel{=}\mathrm{7}{}\<[E]%
\ColumnHook
\end{hscode}\resethooks
Let's now assume that I don't have one, but two favourite
numbers. Curry as a hybrid of functional and logic programming language
allows multiple rules for function definitions.
\begin{hscode}\SaveRestoreHook
\column{B}{@{}>{\hspre}l<{\hspost}@{}}%
\column{E}{@{}>{\hspre}l<{\hspost}@{}}%
\>[B]{}\Varid{favouriteNumber}\;\Conid{Sandra}\mathrel{=}\mathrm{3}{}\<[E]%
\\
\>[B]{}\Varid{favouriteNumber}\;\Conid{Sandra}\mathrel{=}\mathrm{7}{}\<[E]%
\ColumnHook
\end{hscode}\resethooks
This function is non-deterministic because it returns different values
for the same input. The pattern of this function overlaps in
functional programming languages, but Curry's ability to search for
results allows to define those non-deterministic functions.\\

% After that explain how logical variables work.\\
% Give an example request with free variables. Mention the similarity
% to Prolog.
Curry also offers logical variables. A variable is called logical, if
it appears on the right-hand side but not on the left-hand side of a
rule. Free variables are unbound values, that are instantiated to
evaluate an expression. It is possible, that it exists more than one
binding, since Curry computes all possible solutions of an expression.

% As connection to free variables, mention the evaluation methods
% ``narrowing'' and ``residuation'' next. Give an example with boolean and
% constraint equality. Connect this to flexible and rigid functions.\\

Curry provides two different approaches to evaluate an expression with
logical variables. The first approach suspends the evaluation in hopes
that the logical variable will be bound due to another evaluation of
an expression. If there is no other expression to bind the value, the
evaluation fails.This approach is called residuation and Curry uses it
for boolean operators like the boolean equality \ensuremath{==}.  The second
approach, called narrowing, guesses a value for an unbound
value. Constraint operators like the boolean constraint \ensuremath{\mathbin{=.=}} use
narrowing for evaluation. In this context Curry distinguishes two
types of operators: flexible operators that narrow and rigid operators
that use residuation. For example arithmetic (i.e. \ensuremath{\mathbin{+}}, \ensuremath{\mathbin{-}}, \ensuremath{\mathbin{*}} etc)
and other primitive operations are rigid. However these distinctions
do not have any significance for expressions without logical
variables, so called \emph{ground expressions}. As mentioned in the
previous section, Curry evaluates ground expressions with lazy
strategy.\\

Simply put, one can say that Curry combines the features of the
functional programming language Haskell and the logic programming
language Prolog.

% \begin{itemize}
% \item Non-determinism
%   \begin{itemize}
%   \item choose
%   \item vs flexible functions
%   \item not in IO!
%   \end{itemize}
% \end{itemize}
\section{Currydoc}\label{preliminaries:currydoc}
% Explain the idea of CurryDoc, mention the similarity to the known
% javadoc. Current status of the tool (HTML and LaTeX export).\\

CurryDoc is a tool to generate documentation for a program written in
Curry. The current version can generate either a HTML or \LaTeX~ file
as output. CurryDoc works similar to code generating tools like
javadoc\cite{javadoc} as it uses the comments in the source code, which
are provided by the user, to gain information about the function or
data structure. It also provides the type signatures of functions,
since Curry uses a type inferencer algorithm. In addition the CurryDoc tool
analyzes the program's structure and approximates the run-time
behavior to gain further information\cite{currydoc2}. This includes information about
in-/completeness, overlapping pattern matches and non-/deterministic
functions.

Since CurryDoc is implemented in Curry, it uses the meta-programming
module \emph{Flat}\cite{flat2} that provides an intermediate language of the Curry
program to analyze the special function proporties. Such a
FlatCurry\cite{flat} program consists mainly of a list of functions, a
list of types and information about the module itself.
% As you may notice, these are the informations we want to provide for
% the API search engine.

% Introduce FlatCurry program as utility. What is FlatCurry? Give an example of a
% Curry program and the corresponding FlatCurry program.\\

\section{The Holumbus Framework}\label{preliminaries:holumbus}
The Holumbus Framework is a Haskell library created by students of FH
Wedel in connection with three master's theses\footnote{references
  here?}. The main idea of the framework is to collect data with a
specific structure, like an API of a programming language, and to take
advantage of this structure to improve the search results. In addition
to the framework, they also build an example application named
\emph{Hayoo!}\cite{hayoo}, an API search engine for the functional
programming language Haskell.\\

The framework supports three steps to create a search engine: the
crawling, the indexing and the searching part itself. The search
engine produced in connection with this thesis uses the latter two steps. The
indexer preprocesses the documents to create the characteristical data
structure that is used to process a search query. Furthermore Holumbus
provides a data structure that represents the results of a search
query. This data structure corresponds to the structure of the
documents, which simplifies the further processing.\\

\chapter{Analysis}\label{analysis}
This chapter looks into the requirements to build and run Curr(y)gle, an API search
engine for Curry. The first section deals with the creation of an
index, whereas the second section addresses the process of searching
for a query. In this context, we take a closer look of the Holumbus
framework and its features related to searching and evaluate the criteria
to accomplish a user-friendly search mechanism. Above all we need a web
application to handle user queries, which is analyzed in the last
section.

% Give a short summary about the following sections.\\

% What do we need to create this search engine? 
% The possibility to generate and update the index.\\
% The possibility to search for these informations in the index. At
% best: user-friendly \\

% What do we want above all?
% A web application for queries.

% \section{Currydoc}
\section{Creating the Index}
At first, we have to decide about the information we want to store in
the index. Secondly the data structure of the index and documents
provided by the Holumbus framework, that hold
these information, is presented.\\

The data we want to extract from a given Curry module includes at least
the list of defined functions and data structures. We also want to
consider the name, corresponding module and description of these
functions and data structures. Additionally general information about
the module like its name and author needs to be stored. Usually a web
crawler is applied to browse the world wide web for our data. But since Curry is
currently organized by the module documentation generated by CurryDoc,
we already have a mechanism to gain these information about a Curry
module. In fact, we even have more function related information as we
stated in the \hyperref[preliminaries:curryInfo]{previous chapter}. We
know if a function definition is non-deterministic or deterministic
and if a given function is flexible or rigid.

CurryDoc processes Curry modules and generates documentation as HTML
or \LaTeX{} output. For our index, we are not interested in any
document markup language, but the pure information about the Curry
module. This leads to the idea of generating a new readable data
structure as an extension to CurryDoc. In the process we take
advantage of the FlatCurry representation of a Curry module to access
these information we mentioned above. We discuss the actual
implementation in \hyperref[implementation:currydoc]{Section \ref{implementation:currydoc}}. But in
preparation of the next chapter, we introduce the data structure
\ensuremath{\Conid{TypeExpr}} that is provided by the Flat module (see
\hyperref[fig:typeExpr]{Figure \ref{fig:typeExpr}}).

\begin{figure}[h!]
\begin{hscode}\SaveRestoreHook
\column{B}{@{}>{\hspre}l<{\hspost}@{}}%
\column{4}{@{}>{\hspre}l<{\hspost}@{}}%
\column{6}{@{}>{\hspre}l<{\hspost}@{}}%
\column{35}{@{}>{\hspre}l<{\hspost}@{}}%
\column{37}{@{}>{\hspre}l<{\hspost}@{}}%
\column{E}{@{}>{\hspre}l<{\hspost}@{}}%
\>[B]{}\mathbf{data}\;\Conid{TypeExpr}\mathrel{=}{}\<[E]%
\\
\>[B]{}\hsindent{6}{}\<[6]%
\>[6]{}\Conid{TVar}\;\Conid{TVarIndex}{}\<[37]%
\>[37]{}\mbox{\onelinecomment  type variable}{}\<[E]%
\\
\>[B]{}\hsindent{4}{}\<[4]%
\>[4]{}\mid \Conid{FuncType}\;\Conid{TypeExpr}\;\Conid{TypeExpr}{}\<[37]%
\>[37]{}\mbox{\onelinecomment  function type t1->t2}{}\<[E]%
\\
\>[B]{}\hsindent{4}{}\<[4]%
\>[4]{}\mid \Conid{TCons}\;\Conid{QName}\;[\mskip1.5mu \Conid{TypeExpr}\mskip1.5mu]{}\<[37]%
\>[37]{}\mbox{\onelinecomment  type constructor application}{}\<[E]%
\\
\>[37]{}\mbox{\onelinecomment  TCons (module, name) typeargs}{}\<[E]%
\\[\blanklineskip]%
\>[B]{}\mathbf{type}\;\Conid{QName}\mathrel{=}(\Conid{String},\Conid{String}){}\<[35]%
\>[35]{}\mbox{\onelinecomment  (module, type or constructor  name)}{}\<[E]%
\\
\>[B]{}\mathbf{type}\;\Conid{TVarIndex}\mathrel{=}\Conid{Int}{}\<[E]%
\ColumnHook
\end{hscode}\resethooks
\caption{FlatCurry data structure for type signatures}
\label{fig:typeExpr}
\end{figure}

\ensuremath{\Conid{TypeExpr}} represents type and function signatures. It consists of
three constructors to distinguish between a function type (\ensuremath{\Conid{FuncType}\;\Conid{TypeExpr}\;\Conid{TypeExpr}}), a type variable (\ensuremath{\Conid{TVar}\;\Conid{TVarIndex}}), and a type
constructor application (\ensuremath{\Conid{TCons}\;\Conid{QName}\;[\mskip1.5mu \Conid{TypeExpr}\mskip1.5mu]}). An unary type like
\ensuremath{\Conid{Bool}} is represented as a type constructor with an empty list,
i.e. without an application to type arguments. The following code
shows some signatures and their representation in the \ensuremath{\Conid{TypeExpr}} data
structure.

\begin{figure}[h!]
\begin{hscode}\SaveRestoreHook
\column{B}{@{}>{\hspre}l<{\hspost}@{}}%
\column{13}{@{}>{\hspre}l<{\hspost}@{}}%
\column{E}{@{}>{\hspre}l<{\hspost}@{}}%
\>[B]{}\Conid{TCons}\;(\Conid{Prelude},\Conid{Maybe})\;[\mskip1.5mu (\Conid{TCons}\;(\Conid{Prelude},\Conid{Int})\;[\mskip1.5mu \mskip1.5mu]),{}\<[E]%
\\
\>[B]{}\hsindent{13}{}\<[13]%
\>[13]{}(\Conid{TCons}\;(\Conid{Prelude},[\mskip1.5mu \mskip1.5mu])\;[\mskip1.5mu (\Conid{TCons}\;(\Conid{Prelude},\Conid{Char}))\mskip1.5mu])\mskip1.5mu]{}\<[E]%
\ColumnHook
\end{hscode}\resethooks
\caption{The representation of \emph{Maybe Int String} as \ensuremath{\Conid{TypeExpr}}}
\end{figure}
\begin{figure}[h!]
\begin{hscode}\SaveRestoreHook
\column{B}{@{}>{\hspre}l<{\hspost}@{}}%
\column{E}{@{}>{\hspre}l<{\hspost}@{}}%
\>[B]{}\Conid{FuncType}\;(\Conid{TCons}\;(\Conid{Prelude},\Conid{IO})\;[\mskip1.5mu \Conid{TVar}\;\mathrm{97}\mskip1.5mu])\;(\Conid{TCons}\;(\Conid{Prelude},\Conid{IO}){}\<[E]%
\ColumnHook
\end{hscode}\resethooks
\caption{The representation of \emph{IO a -> IO a} as \ensuremath{\Conid{TypeExpr}}}
\end{figure}
\begin{figure}[h!]
\begin{hscode}\SaveRestoreHook
\column{B}{@{}>{\hspre}l<{\hspost}@{}}%
\column{3}{@{}>{\hspre}l<{\hspost}@{}}%
\column{E}{@{}>{\hspre}l<{\hspost}@{}}%
\>[B]{}\Conid{FuncType}\;(\Conid{TCons}\;(\Conid{Prelude},\Conid{Bool})\;[\mskip1.5mu \mskip1.5mu])\;(\Conid{FuncType}\;(\Conid{TCons}\;(\Conid{Prelude},\Conid{Int})\;[\mskip1.5mu \mskip1.5mu]){}\<[E]%
\\
\>[B]{}\hsindent{3}{}\<[3]%
\>[3]{}(\Conid{TCons}\;(\Conid{Prelude},\Conid{Int})\;[\mskip1.5mu \mskip1.5mu])){}\<[E]%
\ColumnHook
\end{hscode}\resethooks
\caption{The representation of \emph{Bool -> Int -> Int} as \ensuremath{\Conid{TypeExpr}}}
\end{figure}

% CurryDoc uses the meta-programming language FlatCurry to gain an
% intermediate data structure. We can use this data structure for our
% purposes. But at first, we have to discuss the which information to
% provide in our data structures. We already introduced \emph{CurryInfo}
% as structure for a Curry module in the
% \hyperref[preliminaries:currydoc:curryInfo]{previous chapter}.
% As next step we want to describe the information this data
% structure holds. The information about a module covers its name,
% author and description.

% Thanks to FlatCurry, we get quite more details about functions and
% types. Besides basic information like the name, description and the
% corresponding module, we have access to function characteristics like
% its flexible or rigid status and non- or deterministic property. In addition
% FlatCurry provides a data structure |TypeExpr| to describe type
% signatures (see \hyperref[fig:typeExpr]{Figure \ref{fig:typeExpr}}),
% which we use in both data structures. All described data structures to
% provide a representation for a Curry module can be seen in
% \hyperref[fig:curryInfo]{Figure \ref{fig:curryInfo}}.

% \begin{figure}[h]
% \begin{code}
% -- || The CurryInfo data holds information about the module, and
% --  corresponding  functions, data and type declaration of a given 
% --  Curry module.
% data CurryInfo = 
%   CurryInfo ModuleInfo [FunctionInfo] [TypeInfo]


% -- || ModuleInfo holds information about the name, author,
% --    and the description of a given module.
% data ModuleInfo = 
%   ModuleInfo String String String

% -- || FunctionInfo holds information about the name, signature, corresponding 
% -- module, description and flexible/rigid status of a function and 
% -- its non-/deterministic behaviour.
% data FunctionInfo = 
%   FunctionInfo String (QName, TypeExpr) String String Bool
%   FlexRigidResult

% -- || TypeInfo holds information about the name, signature, type variables,
% -- corresponding module, and description of a given type.
% data TypeInfo = 
%   TypeInfo String [(QName, [TypeExpr])] [Int] String String
% \end{code}
% \caption{The data structures for a Curry module}
% \label{fig:curryInfo}
% \end{figure}


% What do we need to create an index that can be used for the Curry
% search engine?\\

% First: Curry specific information (CurryDoc)\\

% First start with the idea of the extension: instead of generating a
% document markup language, generate a readable data structure.\\

% Introduce the TypeExpr data structure that is part of the FlatCurry
% feature.\\

% Second: a data structure to hold the information (index and document
% structure from the Holumbus framework)\\
After we decide about the contents of the index, we need to discuss
the storage of these information. We make use of the Holumbus
framework that provides data structures to manage the collected
data. The main idea is to use two structures to score the data.
\ensuremath{\Conid{HolDocuments}\;\Varid{a}} stores the collected data, where \ensuremath{\Varid{a}} is the type of
the document. Secondly the framework provides a data structure
\ensuremath{\Conid{HolIndex}} for the actual index that is traversed in the search
process. Simply put, the \ensuremath{\Conid{HolIndex}} stores pairs \ensuremath{(\Conid{String},\Conid{String})},
where the second entry is the word that can be searched and the first
entry is the context of this word. An example for our API search
engine is a context \emph{description}, whose corresponding \ensuremath{\Conid{String}}
is the description text of a Curry module. This design allows the
association between the information in the index and the corresponding
document through a mapping. A further structure provided by the
framework is \ensuremath{\Conid{HolumbusState}\;\Varid{a}}: a combination of index and document,
polymorph by the data \ensuremath{\Conid{HolDocuments}\;\Varid{a}} holds. In
\hyperref[implementation:index]{Section
  \ref{implementation:index}} we illustrate the use of these data
structures in our implementation.\\

To sum up, we want to extend the current CurryDoc implementation to
generate a new readable data structure about a given Curry
module. These information cover, among other things, function and data
structure definitions, user comments and description in the source
code, and general information about the module. This new data
structure determines the type of \ensuremath{\Conid{HolDocuments}\;\Varid{a}} and forms one part
of the index-document-duo that Holumbus provides. 
% \begin{itemize}
% \item HolDocuments - Stores the documents that correspond to the
%   index. A mapping is provided.
% \item HolIndex - Data structure to store the information, that is
%   traversed in the search process.
% \item HolumbusState a - the combination of index and document,
%   polymorph by the data the HolDocuments holds.
% \end{itemize}

\section{Searching}
% How do we search for the information in the index?
After creating the skeleton for the index and its storage, we want to
actually use these information in a search query. Thus the first step
is to think about the structure of a query, in the second
step we process the query and last we need a representation of the
results of the processed query for further use.\\

Thankfully these are all features the Holumbus framework provides. At
first we take a look at the search mechanism. The data structure
\ensuremath{\Conid{Query}} (see \hyperref[analysis:query]{Figure \ref{analysis:query}})
allows to search for a word and a phrase, both case-insensitive and
case-sensitive. Since the search depends on user-input, the framework
also allows something called \emph{fuzzy search} to scan for results
with spelling errors like transposed letters. Since the index data
structure of Holumbus uses pairs of words and contexts, a special
mechanism to search for these contexts is given. Furthermore the
structure provides binary operators like \ensuremath{\Conid{AND}}, \ensuremath{\Conid{OR}} and \ensuremath{\Conid{NOT}} to
combine multiple queries.
% Holumbus provides search mechanism with a special syntax.
\begin{figure}
\label{analysis:query}
\begin{hscode}\SaveRestoreHook
\column{B}{@{}>{\hspre}l<{\hspost}@{}}%
\column{12}{@{}>{\hspre}l<{\hspost}@{}}%
\column{E}{@{}>{\hspre}l<{\hspost}@{}}%
\>[B]{}\mathbf{data}\;\Conid{Query}\mathrel{=}\Conid{Word}\;\Conid{String}{}\<[E]%
\\
\>[B]{}\hsindent{12}{}\<[12]%
\>[12]{}\mid \Conid{Phrase}\;\Conid{String}{}\<[E]%
\\
\>[B]{}\hsindent{12}{}\<[12]%
\>[12]{}\mid \Conid{CaseWord}\;\Conid{String}{}\<[E]%
\\
\>[B]{}\hsindent{12}{}\<[12]%
\>[12]{}\mid \Conid{CasePhrase}\;\Conid{String}{}\<[E]%
\\
\>[B]{}\hsindent{12}{}\<[12]%
\>[12]{}\mid \Conid{FuzzyWord}\;\Conid{String}{}\<[E]%
\\
\>[B]{}\hsindent{12}{}\<[12]%
\>[12]{}\mid \Conid{Specifier}\;[\mskip1.5mu \Conid{Context}\mskip1.5mu]\;\Conid{Query}{}\<[E]%
\\
\>[B]{}\hsindent{12}{}\<[12]%
\>[12]{}\mid \Conid{Negation}\;\Conid{Query}{}\<[E]%
\\
\>[B]{}\hsindent{12}{}\<[12]%
\>[12]{}\mid \Conid{BinQuery}\;\Conid{BinOp}\;\Conid{Query}\;\Conid{Query}{}\<[E]%
\\
\>[B]{}\mathbf{data}\;\Conid{BinOp}\mathrel{=}\Conid{And}\mid \Conid{Or}\mid \Conid{But}{}\<[E]%
\ColumnHook
\end{hscode}\resethooks
\caption{The \ensuremath{\Conid{Query}} data structure provided by the Holumbus
  framework}
\end{figure}

% And this data structure can be processed by processQuery (Holumbus.Query.Processor).
As next step, we pass the index, document and query to the function
\emph{processQuery} that, as the name suggests, processes the
query. When processing the query, Holumbus only matches for prefixes
of the given word or phrase in a query, we need to consider this
restriction when creating our index in
\hyperref[implementation:index]{Section
  \ref{implementation:index}}. The data structure we get as return
value is shown in \hyperref[analysis:result]{Figure
  \ref{analysis:result}}. Since we make use of this structure in our
implementation, let's take a closer look of the code. \ensuremath{\Conid{Result}\;\Varid{a}}
consists of the matching documents with type \ensuremath{\Varid{a}} as well as possible
word completions. The first is represented by \ensuremath{\Conid{DocHits}\;\Varid{a}} that is a
mapping of \ensuremath{\Conid{DocInfo}\;\Varid{a}} and the contexts. On the other hand \ensuremath{\Conid{DocInfo}\;\Varid{a}}
consists of the matching document and a score. By default this score
is calculated by the number of occurrences of the search query in the
document. But Holumbus also provides a mechanism to apply a customized
ranking function to calculate the score. \ensuremath{\Conid{WordHits}} illustrates the
word completions and is represented by a mapping of the possible completions
of the given prefix in combination with its score, i.e. \ensuremath{\Conid{WordInfo}}, and the contexts.\\
% Holumbus also provides a data structure that is returned after a query

\begin{figure}[h]
\begin{hscode}\SaveRestoreHook
\column{B}{@{}>{\hspre}l<{\hspost}@{}}%
\column{16}{@{}>{\hspre}l<{\hspost}@{}}%
\column{17}{@{}>{\hspre}l<{\hspost}@{}}%
\column{22}{@{}>{\hspre}l<{\hspost}@{}}%
\column{24}{@{}>{\hspre}l<{\hspost}@{}}%
\column{28}{@{}>{\hspre}l<{\hspost}@{}}%
\column{29}{@{}>{\hspre}l<{\hspost}@{}}%
\column{E}{@{}>{\hspre}l<{\hspost}@{}}%
\>[B]{}\mathbf{data}\;\Conid{Result}\;\Varid{a}{}\<[16]%
\>[16]{}\mathrel{=}\Conid{Result}{}\<[E]%
\\
\>[16]{}\hsindent{1}{}\<[17]%
\>[17]{}\{\mskip1.5mu \Varid{docHits}{}\<[28]%
\>[28]{}\mathbin{::}(\Conid{DocHits}\;\Varid{a}){}\<[E]%
\\
\>[16]{}\hsindent{1}{}\<[17]%
\>[17]{},\Varid{wordHits}\mathbin{::}\Conid{WordHits}{}\<[E]%
\\
\>[16]{}\hsindent{1}{}\<[17]%
\>[17]{}\mskip1.5mu\}{}\<[E]%
\\[\blanklineskip]%
\>[B]{}\mathbf{data}\;\Conid{DocInfo}\;\Varid{a}\mathrel{=}\Conid{DocInfo}{}\<[E]%
\\
\>[B]{}\hsindent{17}{}\<[17]%
\>[17]{}\{\mskip1.5mu \Varid{document}\mathbin{::}(\Conid{Document}\;\Varid{a}){}\<[E]%
\\
\>[B]{}\hsindent{17}{}\<[17]%
\>[17]{},\Varid{docScore}\mathbin{::}\Conid{Score}{}\<[E]%
\\
\>[B]{}\hsindent{17}{}\<[17]%
\>[17]{}\mskip1.5mu\}{}\<[E]%
\\[\blanklineskip]%
\>[B]{}\mathbf{data}\;\Conid{WordInfo}{}\<[16]%
\>[16]{}\mathrel{=}\Conid{WordInfo}{}\<[E]%
\\
\>[16]{}\hsindent{1}{}\<[17]%
\>[17]{}\{\mskip1.5mu \Varid{terms}{}\<[29]%
\>[29]{}\mathbin{::}\Conid{Terms}{}\<[E]%
\\
\>[16]{}\hsindent{1}{}\<[17]%
\>[17]{},\Varid{wordScore}\mathbin{::}\Conid{Score}{}\<[E]%
\\
\>[16]{}\hsindent{1}{}\<[17]%
\>[17]{}\mskip1.5mu\}{}\<[E]%
\\[\blanklineskip]%
\>[B]{}\mathbf{type}\;\Conid{DocHits}\;\Varid{a}{}\<[24]%
\>[24]{}\mathrel{=}\Conid{DocIdMap}\;(\Conid{DocInfo}\;\Varid{a},\Conid{DocContextHits}){}\<[E]%
\\[\blanklineskip]%
\>[B]{}\mathbf{type}\;\Conid{DocContextHits}{}\<[24]%
\>[24]{}\mathrel{=}\Conid{Map}\;\Conid{Context}\;\Conid{DocWordHits}{}\<[E]%
\\[\blanklineskip]%
\>[B]{}\mathbf{type}\;\Conid{DocWordHits}{}\<[24]%
\>[24]{}\mathrel{=}\Conid{Map}\;\Conid{Word}\;\Conid{Positions}{}\<[E]%
\\[\blanklineskip]%
\>[B]{}\mathbf{type}\;\Conid{WordHits}{}\<[24]%
\>[24]{}\mathrel{=}\Conid{Map}\;\Conid{Word}\;(\Conid{WordInfo},\Conid{WordContextHits}){}\<[E]%
\\[\blanklineskip]%
\>[B]{}\mathbf{type}\;\Conid{WordContextHits}{}\<[24]%
\>[24]{}\mathrel{=}\Conid{Map}\;\Conid{Context}\;\Conid{WordDocHits}{}\<[E]%
\\[\blanklineskip]%
\>[B]{}\mathbf{type}\;\Conid{WordDocHits}{}\<[24]%
\>[24]{}\mathrel{=}\Conid{Occurrences}{}\<[E]%
\\[\blanklineskip]%
\>[B]{}\mathbf{type}\;\Conid{Score}{}\<[22]%
\>[22]{}\mathrel{=}\Conid{Float}{}\<[E]%
\\
\>[B]{}\mathbf{type}\;\Conid{Terms}{}\<[22]%
\>[22]{}\mathrel{=}[\mskip1.5mu \Conid{String}\mskip1.5mu]{}\<[E]%
\ColumnHook
\end{hscode}\resethooks
\caption{The \ensuremath{\Conid{Result}} data structure provided by the Holumbus
  framework}
\label{analysis:result}
\end{figure}

Summing up, we have discussed the mechanism to evaluate a query with
the Holumbus framework. This includes the data structures to represent a
query, which can be processed to a data structure consisting of the
matching documents and possible word completions.
% But first the user input has to be parsed into the query structure to
% start the processing.

\section{Parsing User-Queries}
% Which criteria do we want to search for? Modules, functions, types,
% signatures, and  det./non-det., flexible/rigid functions. \\
The next question is how to construct the query for a given
user-input. At first we have to decide about the criteria users can
search for. Since the index provides the pairs of contexts and search
words, we are able to restrict the search to all these context with
the help of the \ensuremath{\Conid{Query}} data structure and the \ensuremath{\Conid{Specifier}}
constructor. This allows us to search for modules, functions,
types, signatures and all other contexts we use during the creation of
the index.\\

% First describe the idea, that the use of a specific language increases
% the usability. But it also restricts the user in her usage of the
% search engine, if this language gets more complex. So this results in
% a compromise between a simply to use language and a language that can
% be parsed.  Show the example of searching IO, where the restriction to
% modules minimizes the result.\\
The search mechanism as part of the user-experience is supposed to be
as simple as possible. The use of a specific language increases the
usability, since an expression has its explicit syntax. A good example
is the search term \emph{io}, since in Curry \emph{IO} is the name for
a module, a type and a constructor. Furthermore there are many
functions in the IO module, that contain the word \emph{io}. That
means that the search for \emph{io} results in a great amount of
hits. To reduce the number of hits, we can restrict the search to
a specific context. Therefor we want to provide specifiers in
combination with the term you want to search for, for example
\emph{:function IO} searches for \emph{IO} in the context of function
names only. But this special syntax restricts the user in the use of
the search engine, if the language gets more complex. Thus to provide
a user-friendly search engine, we have to make a compromise between a
simply to use language and language that can be parsed.\\

% Set the focus on signatures. Because Hayoo! does not find signatures with
% redundant parentheses, Curr(y)gle supports parenthesized signatures
% and parenthesized query parts in general. \\
Besides these specifiers we want to parse type signatures of Curry
functions and data structures. Since Hayoo! is not able to parse
redundant parenthesized signatures, we want to address this problem
with great care. Let's assume a beginner searches for a function with
the type signature \emph{IO -> (IO Int)}. The type \emph{IO Int} and
type constructors in general do not need parentheses, but as beginner
you might think they do. Thus we want to support parenthesized
signatures and parenthesized query parts in general.\\

Last but not we want to provide binary conjunctions like \emph{And},
\emph{Or} and \emph{Not}. On the one hand a combination of more search
words is desirable, because popular search engines like
Google\texttrademark~ use this feature. This increases the probability
that users assume binary operations are standard features and expect
search engines to provide the conjunction of several search terms. On
the other hand if the desired result is still vague, a combination of
more search words by a disjunction \emph{Or} helps to narrow down the
search results.\\
% In addition: binary operations/conjunctions. \\
% Explain that in most cases, a combination of more search words is
% desirable, because first popular search engines like Google\texttrademark~use this
% feature so it's common knowledge (the user expects this features) and
% second it's easier to search for more search words, if the desired
% result is still vague.

In the end we want to provide an intuitive but powerful search
engine. With specifiers to restrict the search results to a given
context and with binary operations to narrow down or extend the
contexts, we want to provide a simple language for the user
queries. Additionally type signatures should be recognized, this
includes function, construction and primitive types as well as
redundant parenthesized signatures. As the number of the supported
features increases, the query gets more complex to read. Thus to reach
this goal, we need to analyse the user input and rebuild it as an
expression of our \ensuremath{\Conid{Query}} data structure. In
\hyperref[implementation:parser]{Section \ref{implementation:parser}}
we discuss how we implement the construction of the \ensuremath{\Conid{Query}} structure
for a given user-input.

% The parser becomes a complex, but very important matter.
\chapter{Implementation}\label{implementation}
Mention that the implementation is done in Haskell or Curry.\\

Give a short summary of the next sections.
\section{CurryDoc extension}\label{implementation:currydoc}

% First start with the idea of the extension: instead of generating a
% document markup language, generate a readable data structure.

% Introduce the TypeExpr data structure that is part of the FlatCurry
% feature.

% \begin{code}
% data TypeExpr =
%      TVar TVarIndex                 -- type variable
%    | FuncType TypeExpr TypeExpr     -- function type t1->t2
%    | TCons QName [TypeExpr]         -- type constructor application
%                                     -- TCons (module, name) typeargs

% type QName = (String, String)
% type TVarIndex = Int
% \end{code}

Present the general structure of the \ensuremath{\Conid{CurryInfo}} data and the
sub-structures \ensuremath{\Conid{ModuleInfo}}, \ensuremath{\Conid{FunctionInfo}} and \ensuremath{\Conid{TypeInfo}}.

\begin{hscode}\SaveRestoreHook
\column{B}{@{}>{\hspre}l<{\hspost}@{}}%
\column{E}{@{}>{\hspre}l<{\hspost}@{}}%
\>[B]{}\mathbf{data}\;\Conid{CurryInfo}\mathrel{=}\Conid{CurryInfo}\;\Conid{ModuleInfo}\;[\mskip1.5mu \Conid{FunctionInfo}\mskip1.5mu]\;[\mskip1.5mu \Conid{TypeInfo}\mskip1.5mu]{}\<[E]%
\ColumnHook
\end{hscode}\resethooks

The name, author and description is the interesting information that
is exported. 
\begin{hscode}\SaveRestoreHook
\column{B}{@{}>{\hspre}l<{\hspost}@{}}%
\column{E}{@{}>{\hspre}l<{\hspost}@{}}%
\>[B]{}\mathbf{data}\;\Conid{ModuleInfo}\mathrel{=}\Conid{ModuleInfo}\;\Conid{String}\;\Conid{String}\;\Conid{String}\;[\mskip1.5mu \Conid{String}\mskip1.5mu]\;\Conid{String}{}\<[E]%
\ColumnHook
\end{hscode}\resethooks

The model contains the name, signature, module, and description of a function
and extra information about the non-/determinism and flexible/rigid status.

\begin{hscode}\SaveRestoreHook
\column{B}{@{}>{\hspre}l<{\hspost}@{}}%
\column{3}{@{}>{\hspre}l<{\hspost}@{}}%
\column{E}{@{}>{\hspre}l<{\hspost}@{}}%
\>[B]{}\mathbf{data}\;\Conid{FunctionInfo}\mathrel{=}\Conid{FunctionInfo}\;\Conid{String}\;(\Conid{QName},\Conid{TypeExpr})\;\Conid{String}{}\<[E]%
\\
\>[B]{}\hsindent{3}{}\<[3]%
\>[3]{}\Conid{String}\;\Conid{Bool}\;\Conid{FlexRigidResult}{}\<[E]%
\ColumnHook
\end{hscode}\resethooks

The TypeInfo includes the name, signature, list of type variables,
module, and description.
\begin{hscode}\SaveRestoreHook
\column{B}{@{}>{\hspre}l<{\hspost}@{}}%
\column{3}{@{}>{\hspre}l<{\hspost}@{}}%
\column{E}{@{}>{\hspre}l<{\hspost}@{}}%
\>[B]{}\mathbf{data}\;\Conid{TypeInfo}\mathrel{=}\Conid{TypeInfo}\;\Conid{String}\;[\mskip1.5mu (\Conid{QName},[\mskip1.5mu \Conid{TypeExpr}\mskip1.5mu])\mskip1.5mu]\;[\mskip1.5mu \Conid{TVarIndex}\mskip1.5mu]{}\<[E]%
\\
\>[B]{}\hsindent{3}{}\<[3]%
\>[3]{}\Conid{String}\;\Conid{String}{}\<[E]%
\ColumnHook
\end{hscode}\resethooks

Explain how this extension is used in the CurryDoc tool. 

The same data structure is used on the Haskell side that implements
the search engine.

\section{Indexing}\label{implementation:index}

First mention that the interessting parts (that we want to add to the
index) are already filtered by the CurryDoc part. So the indexer only
processes the information to the structure provided by the Holumbus
framework (HolumbusState).

Present the output (files) the indexer produces. After that introduce
the concept of documents and index. In addition to that, say something
about refreshing and checking the list of modules.

Mention the three different kinds of documents for the three
structures: module, function, type.
\begin{hscode}\SaveRestoreHook
\column{B}{@{}>{\hspre}l<{\hspost}@{}}%
\column{35}{@{}>{\hspre}l<{\hspost}@{}}%
\column{E}{@{}>{\hspre}l<{\hspost}@{}}%
\>[B]{}\mbox{\onelinecomment  | Pair of index and documents of the type ModuleInfo}{}\<[E]%
\\
\>[B]{}\mathbf{type}\;\Conid{CurryModIndexerState}{}\<[35]%
\>[35]{}\mathrel{=}\Conid{HolumbusState}\;\Conid{ModuleInfo}{}\<[E]%
\\[\blanklineskip]%
\>[B]{}\mbox{\onelinecomment  | Pair of index and documents of the type FunctionInfo}{}\<[E]%
\\
\>[B]{}\mathbf{type}\;\Conid{CurryFctIndexerState}{}\<[35]%
\>[35]{}\mathrel{=}\Conid{HolumbusState}\;\Conid{FunctionInfo}{}\<[E]%
\\[\blanklineskip]%
\>[B]{}\mbox{\onelinecomment  | Pair of index and documents of the type TypeInfo}{}\<[E]%
\\
\>[B]{}\mathbf{type}\;\Conid{CurryTypeIndexerState}{}\<[35]%
\>[35]{}\mathrel{=}\Conid{HolumbusState}\;\Conid{TypeInfo}{}\<[E]%
\ColumnHook
\end{hscode}\resethooks

Explain how each kind of information (description, module name,
function signature etc) is combined with its context, and that these
are stored in the index. Note that information can be extracted by the
context again.

Note the difficulties of updating the index, because the data
structure of the loaded pair of index and document differs from
HolumbusState a. Conversion of \ensuremath{\Conid{Inverted}} to \ensuremath{\Conid{CompactInverted}} and
\ensuremath{\Conid{Documents}\;\Varid{a}} to \ensuremath{\Conid{SmallDocuments}\;\Varid{a}}.
\begin{hscode}\SaveRestoreHook
\column{B}{@{}>{\hspre}l<{\hspost}@{}}%
\column{E}{@{}>{\hspre}l<{\hspost}@{}}%
\>[B]{}\mathbf{type}\;\Conid{LoadedIndexerState}\;\Varid{a}\mathrel{=}(\Conid{CompactInverted},\Conid{SmallDocuments}\;\Varid{a}){}\<[E]%
\ColumnHook
\end{hscode}\resethooks

Refer to the appendix, where the usage of the curryIndexer is
explained.

\section{Parsing user queries}\label{implementation:parser}

% First describe the idea, that the use of a specific language increases
% the usability. But it also restricts the user in her usage of the
% search engine, if this language gets more complex. So this results in
% a compromise between a simply to use language and a language that can
% be parsed.  Show the example of searching IO, where the restriction to
% modules minimizes the result.

% After that list all language components that describe the restriction
% to contexts (module, function, type, signature and  non-/deterministic,
% flexible, rigid functions.

% Set the focus on signatures. Because Hayoo does not find signatures with
% redundant parenthesis, Curr(y)gle supports parenthesized signatures
% and parenthesized query parts in general.  

Explain the general idea of parsers: a parser is used to analyze the
user query with these different kinds of information. While parsing
the string, a new data structure is composed for further use. This
data structure is provided by the Holumbus framework and is used to
starte the processs that returns the search results.

Introduce the parser type that is parametrized with the type to parse and the
resulting type.

\begin{hscode}\SaveRestoreHook
\column{B}{@{}>{\hspre}l<{\hspost}@{}}%
\column{E}{@{}>{\hspre}l<{\hspost}@{}}%
\>[B]{}\mathbf{type}\;\Conid{Parser}\;\Varid{s}\;\Varid{a}\mathrel{=}[\mskip1.5mu \Varid{s}\mskip1.5mu]\to [\mskip1.5mu (\Varid{a},[\mskip1.5mu \Varid{s}\mskip1.5mu])\mskip1.5mu]{}\<[E]%
\ColumnHook
\end{hscode}\resethooks

Explain the operator \ensuremath{(\mathbin{<|>})} that applies two parsers and concats the
parsing results.  
\begin{hscode}\SaveRestoreHook
\column{B}{@{}>{\hspre}l<{\hspost}@{}}%
\column{E}{@{}>{\hspre}l<{\hspost}@{}}%
\>[B]{}(\mathbin{<|>})\mathbin{::}\Conid{Parser}\;\Varid{s}\;\Varid{a}\to \Conid{Parser}\;\Varid{s}\;\Varid{a}\to \Conid{Parser}\;\Varid{s}\;\Varid{a}{}\<[E]%
\\
\>[B]{}\Varid{p}\mathbin{<|>}\Varid{q}\mathrel{=}(\lambda \Varid{ts}\to \Varid{p}\;\Varid{ts}++\Varid{q}\;\Varid{ts}){}\<[E]%
\ColumnHook
\end{hscode}\resethooks

Introduce the combination of two parsers that is similar to a monadic
bind operator. 
\begin{hscode}\SaveRestoreHook
\column{B}{@{}>{\hspre}l<{\hspost}@{}}%
\column{E}{@{}>{\hspre}l<{\hspost}@{}}%
\>[B]{}(\mathbin{<*>})\mathbin{::}\Conid{Parser}\;\Varid{s}\;(\Varid{a}\to \Varid{b})\to \Conid{Parser}\;\Varid{s}\;\Varid{a}\to \Conid{Parser}\;\Varid{s}\;\Varid{b}{}\<[E]%
\\
\>[B]{}\Varid{p}\mathbin{<*>}\Varid{q}\mathrel{=}\lambda \Varid{ts}\to [\mskip1.5mu (\Varid{f}\;\Varid{x},\Varid{ts2})\mid (\Varid{f},\Varid{ts1})\leftarrow \Varid{p}\;\Varid{ts},(\Varid{x},\Varid{ts2})\leftarrow \Varid{q}\;\Varid{ts1}\mskip1.5mu]{}\<[E]%
\ColumnHook
\end{hscode}\resethooks

Now show that a function has to be applied, to use the binding operator.
\begin{hscode}\SaveRestoreHook
\column{B}{@{}>{\hspre}l<{\hspost}@{}}%
\column{12}{@{}>{\hspre}l<{\hspost}@{}}%
\column{E}{@{}>{\hspre}l<{\hspost}@{}}%
\>[B]{}(\mathbin{<\$>})\mathbin{::}(\Varid{a}\to \Varid{b})\to \Conid{Parser}\;\Varid{s}\;\Varid{a}\to \Conid{Parser}\;\Varid{s}\;\Varid{b}{}\<[E]%
\\
\>[B]{}\Varid{f}\mathbin{<\$>}\Varid{p}\mathrel{=}{}\<[12]%
\>[12]{}\lambda \Varid{ts}\to [\mskip1.5mu (\Varid{f}\;\Varid{x},\Varid{ts1})\mid (\Varid{x},\Varid{ts1})\leftarrow \Varid{p}\;\Varid{ts}\mskip1.5mu]{}\<[E]%
\ColumnHook
\end{hscode}\resethooks

Present a simple example that can parse a given character.
\begin{hscode}\SaveRestoreHook
\column{B}{@{}>{\hspre}l<{\hspost}@{}}%
\column{3}{@{}>{\hspre}l<{\hspost}@{}}%
\column{17}{@{}>{\hspre}l<{\hspost}@{}}%
\column{E}{@{}>{\hspre}l<{\hspost}@{}}%
\>[B]{}\Varid{parsePredicate}\mathbin{::}(\Conid{Char}\to \Conid{Bool})\to \Conid{Parser}\;\Conid{Char}\;\Conid{String}{}\<[E]%
\\
\>[B]{}\Varid{parsePredicate}\;\Varid{predicate}\;(\Varid{t}\mathbin{:}\Varid{ts}){}\<[E]%
\\
\>[B]{}\hsindent{3}{}\<[3]%
\>[3]{}\mid \Varid{predicate}\;\Varid{t}\mathrel{=}[\mskip1.5mu ([\mskip1.5mu \mskip1.5mu],\Varid{ts})\mskip1.5mu]{}\<[E]%
\\
\>[B]{}\hsindent{3}{}\<[3]%
\>[3]{}\mid \Varid{otherwise}{}\<[17]%
\>[17]{}\mathrel{=}[\mskip1.5mu ([\mskip1.5mu \mskip1.5mu],\Varid{t}\mathbin{:}\Varid{ts})\mskip1.5mu]{}\<[E]%
\\
\>[B]{}\Varid{parsePredicate}\;\anonymous \;[\mskip1.5mu \mskip1.5mu]\mathrel{=}[\mskip1.5mu \mskip1.5mu]{}\<[E]%
\ColumnHook
\end{hscode}\resethooks

Present the concept by writing a parser for expressions with or without
parentheses using \ensuremath{\mathbin{<|>}} and \ensuremath{\mathbin{<*>}}. 

\begin{hscode}\SaveRestoreHook
\column{B}{@{}>{\hspre}l<{\hspost}@{}}%
\column{3}{@{}>{\hspre}l<{\hspost}@{}}%
\column{5}{@{}>{\hspre}l<{\hspost}@{}}%
\column{7}{@{}>{\hspre}l<{\hspost}@{}}%
\column{E}{@{}>{\hspre}l<{\hspost}@{}}%
\>[B]{}\mbox{\onelinecomment  Parses an opening parenthesis by using the predicate parser.}{}\<[E]%
\\
\>[B]{}\Varid{parseOpenParenthesis}\mathbin{::}\Conid{Parser}\;\Conid{Char}\;\Conid{String}{}\<[E]%
\\
\>[B]{}\Varid{parseOpenParenthesis}\mathrel{=}{}\<[E]%
\\
\>[B]{}\hsindent{3}{}\<[3]%
\>[3]{}\Varid{parsePredicate}\;(\text{\tt '('}==){}\<[E]%
\\[\blanklineskip]%
\>[B]{}\mbox{\onelinecomment  Parses an closing parenthesis.}{}\<[E]%
\\
\>[B]{}\Varid{parseCloseParenthesis}\mathbin{::}\Conid{Parser}\;\Conid{Char}\;\Conid{String}{}\<[E]%
\\
\>[B]{}\Varid{parseCloseParenthesis}\mathrel{=}{}\<[E]%
\\
\>[B]{}\hsindent{3}{}\<[3]%
\>[3]{}\Varid{parsePredicate}\;(\text{\tt ')'}==){}\<[E]%
\\[\blanklineskip]%
\>[B]{}\mbox{\onelinecomment  Parses a sequence of alpha numeric characters. }{}\<[E]%
\\
\>[B]{}\Varid{parseExpression}\mathbin{::}\Conid{Parser}\;\Conid{Char}\;\Conid{String}{}\<[E]%
\\
\>[B]{}\Varid{parseExpression}\;(\Varid{t}\mathbin{:}\Varid{ts})\mathrel{=}{}\<[E]%
\\
\>[B]{}\hsindent{3}{}\<[3]%
\>[3]{}\mathbf{if}\;\Varid{isAlphaNum}\;\Varid{t}\;\mathbf{then}\;\Varid{parseExpression}\;\Varid{ts}\;\mathbf{else}\;[\mskip1.5mu ([\mskip1.5mu \mskip1.5mu],\Varid{t}\mathbin{:}\Varid{ts})\mskip1.5mu]{}\<[E]%
\\
\>[B]{}\Varid{parseExpression}\;[\mskip1.5mu \mskip1.5mu]\mathrel{=}[\mskip1.5mu \mskip1.5mu]{}\<[E]%
\\[\blanklineskip]%
\>[B]{}\mbox{\onelinecomment  Parses an expression or a parenthesized expression.}{}\<[E]%
\\
\>[B]{}\Varid{parenthesizedExpression}\mathbin{::}\Conid{Parser}\;\Conid{Char}\;\Conid{String}{}\<[E]%
\\
\>[B]{}\Varid{parenthesizedExpression}\mathrel{=}{}\<[E]%
\\
\>[B]{}\hsindent{5}{}\<[5]%
\>[5]{}((\mathbin{\char92 \char95 }\Varid{expr}\;\anonymous \to \Varid{expr}){}\<[E]%
\\
\>[5]{}\hsindent{2}{}\<[7]%
\>[7]{}\mathbin{<\$>}\Varid{parseOpenParenthesis}\mathbin{<*>}\Varid{parseExpression}\mathbin{<*>}\Varid{parseCloseParenthesis}){}\<[E]%
\\
\>[B]{}\hsindent{5}{}\<[5]%
\>[5]{}\mathbin{<|>}\Varid{parseExpression}{}\<[E]%
\ColumnHook
\end{hscode}\resethooks

Mention that the used type in the implementation correlates to \ensuremath{\Conid{Parser}\;\Conid{String}\;\Conid{TypeExpr}} (which is already used for the data exchange) for
signatures and \ensuremath{\Conid{Parser}\;\Conid{String}\;\Conid{Query}} (which is provided by the Holumbus framework) for the end
result. 

% Explain that in most cases, a combination of more search words is
% desirable, because first popular search engines like Google\texttrademark~use this
% feature so it's common knowledge (the user expects this features) and
% second it's easier to search for more search words, if the desired
% result is still vague.

Give a definition of the language (EBNF(?) / appendix). 

\begin{hscode}\SaveRestoreHook
\column{B}{@{}>{\hspre}l<{\hspost}@{}}%
\column{5}{@{}>{\hspre}l<{\hspost}@{}}%
\column{16}{@{}>{\hspre}l<{\hspost}@{}}%
\column{43}{@{}>{\hspre}l<{\hspost}@{}}%
\column{60}{@{}>{\hspre}l<{\hspost}@{}}%
\column{64}{@{}>{\hspre}l<{\hspost}@{}}%
\column{E}{@{}>{\hspre}l<{\hspost}@{}}%
\>[B]{}\Varid{query}\mathbin{:=}\Varid{expr}\;[\mskip1.5mu \Varid{expr}\mskip1.5mu]\mid \Varid{expr}\;\Varid{bool}\;\Varid{expr}\mid (\Varid{expr}){}\<[E]%
\\
\>[B]{}\Varid{expr}\mathbin{:=}(\Varid{expr})\mid \Varid{specifier}\mid \Varid{signature}\mid \Varid{string}{}\<[E]%
\\
\>[B]{}\Varid{bool}\mathbin{:=}``\Conid{AND''}\mid ``\Conid{OR''}\mid ``\Conid{NOT''}{}\<[E]%
\\
\>[B]{}\Varid{specifier}\mathbin{:=}{}\<[16]%
\>[16]{}``\mathbin{:}\Varid{module''}\;[\mskip1.5mu \Varid{alphaNum}\mskip1.5mu]\mid ``\mathbin{:}\Varid{signature''}\;{}\<[60]%
\>[60]{}[\mskip1.5mu \Varid{signature}\mskip1.5mu]{}\<[E]%
\\
\>[B]{}\mid {}\<[5]%
\>[5]{}``\mathbin{:}\Varid{function''}\;[\mskip1.5mu \Varid{alphaNum}\mskip1.5mu]\mid ``\mathbin{:}\Varid{flexible''}\mid ``\mathbin{:}\Varid{rigid''}\mid {}\<[64]%
\>[64]{}``\mathbin{:}\Varid{nondet''}\mid ``\mathbin{:}\Varid{det''}{}\<[E]%
\\
\>[B]{}\Varid{signature}\mathbin{:=}\Conid{Upper}\;\Varid{alphaNum}\mid \Varid{function}\mid \Varid{constructor}{}\<[E]%
\\
\>[B]{}\Varid{function}\mathbin{:=}\Varid{signature}``\mathbin{-->}\text{\tt ''}\;\Varid{signature}{}\<[43]%
\>[43]{}\mid \Varid{lower}``\mathbin{-->}\text{\tt ''}\;\Varid{signature}\mid {}\<[E]%
\\
\>[B]{}\Varid{signature}``\to \text{\tt ''}\;\Varid{lower}{}\<[E]%
\\
\>[B]{}\Varid{constructor}\mathbin{:=}\Conid{Upper}\;\Varid{alphaNum}\;\Varid{signature}\mid \Conid{Upper}\;\Varid{alphaNum}\;\Varid{lower}{}\<[E]%
\ColumnHook
\end{hscode}\resethooks

%%%%%%%%%%%%%%%%%%%%%%%%%%%%%%%%%%%%%%%%%%%%%%%%%%%%%
% The actual use case of the search engine is to search for a given query.

% For the user, it's important that the search engine understands her
% query. For this cause, the user is obliged to use a specific language
% that can be interpreted unambiguously.

% To simplify the usage of the search engine, Curr(y)gle\texttrademark
% provides a language to restrict the search to a specific
% context. This feature not only simplifies the use, but results in a better
% user-experience.

% (Example) Let's assume you want to search for the module Map. Without
% the restriction to modules, there are numerous results, because map is
% a very common name (as it is associated with higher-order functions)
% in functional languages. Thanks to the feature to search for specific
% contexts, you can search for the query \emph{:module map} instead,
% where \emph{:module} indicates the context of modules.

% Curr(y)gle \texttrademark supports to search explicitly for modules,
% functions, types and signatures. In addition to that, it allows
% to search for functions with a given characteristic, like
% non-deterministic, flexible and rigid functions.

% As the number of the supported features increases, the query gets more
% complex to parse. 

\chapter{Conclusion}\label{conclusion}
\section{Summary and Results}
\section{Outlook}

% Setze Numerierung wieder auf römisch zurück und setzte von oben fort
% Wert ist demnach der von 'roemisch'
% \newpage
% \pagenumbering{Roman}
% \setcounter{page}{\value{roemisch}}

% Literaturverzeichnis
\bibliography{literatur/bib}

% Appendix, falls vorhanden
% \appendix
% \input{extras/anhang}

% Eidesstattliche Erklärung
% \input{extras/eidesstattliche}

\end{document}
